\documentclass[1gpt]{article}
\usepackage[a4paper, left=1in, right=1in, top=1in, bottom=1in]{geometry}
\usepackage{amsfonts}
\usepackage{amsmath}
\usepackage[amsthm]{ntheorem}
\usepackage{parskip}
\usepackage{titling}
\usepackage[colorlinks]{hyperref}
\usepackage{pdfcomment}
\setlength{\parindent}{0in}
\setlength{\textheight}{9.in}

\theoremstyle{break}
\theoremindent=2em
\theoremheaderfont{\kern-1em\normalfont\bfseries}
\newtheorem{theorem}{Theorem}[section]
\newtheorem{lemma}[theorem]{Lemma}
\newtheorem{proposition}[theorem]{Proposition}

\newtheorem{definition}[theorem]{Definition}

\setlength{\droptitle}{-4em}
\title{The Pow-XOR Confusion Problem and an Efficient Solution}
\vspace{-2em}
\author{Akshay Trivedi}
\date{\vspace{-3em}}

\newcommand{\xor}{\oplus}
\renewcommand{\familydefault}{\sfdefault}

\begin{document}

\maketitle

\section{Problem Statement}

When writing equations over ASCII text, people often use the `\textasciicircum'
character to denote exponentiation. For example, \texttt{2 {\textasciicircum} 3
+ 4} means $2^3+4$ which evaluates to $12$. Many programming languages use the
same symbol for the bitwise exclusive-or operation, and with this definition
the expression \texttt{(2 {\textasciicircum} 3) + 4} evaluates to 5. For what
values of $a$ and $b$ do these two meanings of `\textasciicircum' produce the
same value for the expression \texttt{a {\textasciicircum} b} (using unsigned
$n$-bit integers)?

\subsection{Formal Problem Statement}

\subsubsection{Preliminary Stuff}

The symbol $\mathbb{N}$ denotes the set of natural numbers, including zero,
i.e. $\mathbb{N} = \{0, 1, 2, 3, \ldots\}$.

The symbol $\mathbb{Z}$ denotes the set of integers, which are the natural
numbers and their negations (additive inverses), i.e. $\mathbb{Z} = \{\ldots,
-2, -1, 0, 1, 2, \ldots\}$.

\begin{definition}[congruence modulo an integer]
    \label{def:mod}
    Let $x, y \in \mathbb{Z},\; n \in \mathbb{N}$. For $x$ and $y$ to be
    congruent modulo $n$, denoted by $x \equiv y \pmod{n}$, means that $x = y +
    k \cdot n$ for some $k \in \mathbb{Z}$. The $n$ is called the modulus.
\end{definition}

\begin{lemma}
    \label{lem:bit-of}
    For every $x \in \mathbb{Z},\; n \in \mathbb{N}$, there exists a unique $d
    \in \{ 0, 1 \}$ such that $x \equiv 2^n \cdot d + x_r \pmod{2^{n+1}}$ for
    some $x_r \in \mathbb{Z},\; 0 \leq x_r < 2^n$.
\end{lemma}

\begin{definition}[bit of an integer]
    \label{def:bit-of}
    For every $x \in \mathbb{Z},\; n \in \mathbb{N}$, the $n^\textrm{th}$ bit
    of $x$ is the $d \in \{ 0, 1 \}$ which satisfies \ref{lem:bit-of}.
\end{definition}

\begin{lemma}
    \label{lem:xor}
    For every $x, y \in \mathbb{Z}$, there exists a unique $w \in \mathbb{Z}$
    such that for all $n \in \mathbb{N}$, the $n^\textrm{th}$ bit of $w$ is 1
    if and only if exactly one of the $n^\textrm{th}$ bits of $x$ and $y$ is 1.
\end{lemma}

\begin{definition}[exclusive-or / XOR]
    \label{def:xor}
    For every $x, y \in \mathbb{Z}$, the exclusive-or/XOR of $x$ and $y$ is the
    integer $w$ which satisfies \ref{lem:xor} and is denoted by $x \xor y = w$.
\end{definition}

\subsubsection{The Problem}

Let $n \in \mathbb{N}$.

\begin{definition}[$X_n$]
    \label{def:xn}
    $X_n$ is the set of all solutions $(a, b) \in \mathbb{Z} \times
    \mathbb{Z},\; 0 \leq a,b < 2^n$ to the congruence $a^b \equiv a \xor b
    \pmod{2^n}$.
\end{definition}

The problem is to design an efficient algorithm which, given $n$, enumerates
all of of $X_n$. Bonus points if, given one or two of the three variables $a$,
$b$, and  $n$, the algorithm can efficiently enumerate all possibilities for
the remaining variables.

\subsection{Examples}

\begin{itemize}

    \item

        For all $n \in \mathbb{N},\; n \geq 1$, $1^0 \equiv 1 \pmod{2^n}$ and
        $1 \xor 0 = 1$, so $(1, 0) \in X_n$.

    \item

        For all $n \in \mathbb{N}, \; n \geq 1$, $(2^n-1)^{2^n-2} \equiv 1
        \pmod{2^n}$ and $(2^n-1) \xor (2^n-2) \equiv 1$, so $(n-1, n-2) \in
        X_n$.

    \item

        $0^0 = 1 \equiv 0 = 0 \xor 0 \pmod{2^0}$, so $(0, 0) \in X_0$ (in fact,
        $X_0 = \{(0, 0)\}$).

    \item

        $(3109287477, 2325659185) \in X_{32}$ because $$3109287477^{869091332}
        \equiv 2325659185 = 3109287477 \xor 869091332 \pmod{2^{32}}.$$

\end{itemize}

\subsection{Properties of \texorpdfstring{$X_n$}{Xn}}

Warning: This section contains minor spoilers for the solution. I thought this
problem was pretty fun, so I encourage you to stop here and try it for yourself
if you like algebra and number theory.

You might expect the size of $X_n$ to be kind of ``random-ish'' and difficult
to compute since modular exponentiation and bitwise exclusive-or come from
different and seemingly unrelated structures on $\mathbb{Z}_{2^n}$. However, as
it turns out,

\begin{proposition}[Strange Fact]
    For all $n \in \mathbb{N}, \; n \geq 1$, $|X_n| = 2^n - 1$.
\end{proposition}

After observing this result, you might expect that each solution $(a, b) \in
X_n$ is uniquely determined by either $a$ or $b$. My initial guess was $a$, as
intuitively I expected the base to dictate more properties of the exponent
operation, like the number of exponents which map to unique powers $\pmod{2^n}$
by Euler's Theorem.

I collected the frequencies of $a$ and $b$ for $(a, b) \in X_n$ (for $n \geq
1$), and was surprised by the results:

\begin{itemize}

    \item

        $b$ is always even. $b = 0$ occurs exactly once, and comes from the
        solution $(a, b) = (1, 0) \in X_n$. Every positive even integer $0 < b
        < 2^n$ occurs exactly twice. By summing the frequencies of $b$'s,
        $|X_n| = 2 \cdot (2^{n-1} - 1) + 1 = 2^n - 1$ since there are two
        solutions for each of the $2^{n-1} - 1$ even positive integers and one
        extra solution for $b=0$.

    \item

        Every odd $a$ occurs exactly once ($2^{n-1}$ solutions with odd $a$).

    \item

        $a=0$ never occurs. \textit{Almost} every positive even $a$ occurs
        once. There are very few exceptions that either occur exactly twice, or
        never occur at all, and these two kind of exceptions are equinumerous
        (excluding $a=0$). Hence there are $2^{n-1}-1$ solutions with even $a$,
        which is what you'd expect.


        \begin{table}[h]
            \centering
            \begin{tabular}{|c|l|l|}
                \hline
                \multicolumn{1}{|c|}{$n$} & \multicolumn{1}{|c|}{$a$ occuring exactly twice (duplicate values)} & \multicolumn{1}{|c|}{nonzero $a$ occuring zero times (absent values)} \\
                \hline
                $1$  &                                                       & \\
                $2$  &                                                       & \\
                $3$  & $6$                                                   & $2$ \\
                $4$  & $6$                                                   & $2$ \\
                $5$  & $6$                                                   & $2$ \\
                $6$  & $38$                                                  & $2$ \\
                $7$  & $38, 70$                                              & $2, 6$ \\
                $8$  & $166, 70$                                             & $2, 6$ \\
                $9$  & $422, 260, 70$                                        & $2, 4, 6$ \\
                $10$ & $934, 260, 582$                                       & $2, 4, 6$ \\
                $11$ & $1958, 260, 1606, 1034$                               & $2, 4, 6, 10$ \\
                $12$ & $4006, 260, 1606, 1034$                               & $2, 4, 6, 10$ \\
                $13$ & $8102, 260, 1606, 1034$                               & $2, 4, 6, 10$ \\
                $14$ & $8102, 260, 1606, 9226$                               & $2, 4, 6, 10$ \\
                $15$ & $8102, 260, 17990, 25610, 16398$                      & $2, 4, 6, 10, 14$ \\
                $16$ & $40870, 260, 50758, 58378, 16398$                     & $2, 4, 6, 10, 14$ \\
                $17$ & $106406, 65796, 50758, 123914, 16398$                 & $2, 4, 6, 10, 14$ \\
                $18$ & $237478, 65796, 181830, 254986, 16398$                & $2, 4, 6, 10, 14$ \\
                $19$ & $237478, 65796, 181830, 254986, 278542, 262162$       & $2, 4, 6, 10, 14, 18$ \\
                $20$ & $237478, 65796, 181830, 779274, 278542, 262162$       & $2, 4, 6, 10, 14, 18$ \\
                $21$ & $1286054, 65796, 1230406, 1827850, 1327118, 262162$   & $2, 4, 6, 10, 14, 18$ \\
                $22$ & $3383206, 2162948, 3327558, 1827850, 1327118, 262162$ & $2, 4, 6, 10, 14, 18$ \\
                \hline
            \end{tabular}
        \end{table}

        \item

            In fact, most solutions with a positive even $a$ satisfy $a=b$. The
            only exceptions are when $a$ is in the left-hand-side column of the
            table above and $b$ is the corresponding value from the
            right-hand-side column. So for example, in row 22, the second
            duplicate $a$ is 2162948, and the second absent $a$ is 4. This
            means $2162948^4 = 2162948 \xor 4 \pmod{2^{22}}$. With the table
            above (and that $a=b$ for the rest of the solutions), you can
            enumerate all solutions $(a, b) \in X_n$ with a positive even $a$
            for $n \leq 22$.

\end{itemize}

\section{Solution}

\subsection{Preliminary Stuff}

The lemmas in this section have really easy proofs, and are either generally
well-known or conveniently rewritten from well-known statements.

\begin{lemma}[congruence modulo an integer is an equivalence relation]
    \label{lem:mod-eqrln}
    For every $n \in \mathbb{N}$, congruence modulo $n$ is an equivalence
    relation (transitive, symmetric, reflexive). In other words, for all $x, y,
    z \in \mathbb{Z}$:
    $$ \displaylines{
        \textrm{if } x \equiv y \pmod{n}
            \textrm{, and } y \equiv z \pmod{n}
            \textrm{ then } x \equiv z \pmod{n} \cr
       \textrm{if } x \equiv y \pmod{n} \textrm{ then } y \equiv x \pmod{n} \cr
        x \equiv x \pmod{n}}
    $$
\end{lemma}

\begin{lemma}[reducing the modulus]
    \label{lem:mod-red} For every $x, y \in \mathbb{Z},\; n, n^\prime \in
    \mathbb{Z}$, if $x \equiv y \pmod{n}$ and $n \equiv 0 \pmod{n^\prime}$,
    then $x \equiv y \pmod{n^\prime}$.
\end{lemma}

\begin{lemma}[congruence modulo an integer respects ring operations]
    \label{lem:mod}
    For every $n \in \mathbb{N}$, congruence modulo $n$ ``respects'' addition,
    subtraction, and multiplication, in the sense that if $x, x^\prime, y,
    y^\prime \in \mathbb{Z}$ such that $x \equiv x^\prime \pmod{n}$ and $y
    \equiv y^\prime \pmod{n}$, then
    $$ x + y \equiv x^\prime + y^\prime \pmod{n} \qquad\qquad
       x - y \equiv x^\prime - y^\prime \pmod{n} \qquad\qquad
       x \cdot y \equiv x^\prime \cdot y^\prime \pmod{n} $$
\end{lemma}

\begin{lemma}
    \label{lem:red-mod}
    For every integer $x \in \mathbb{Z}, n \in \mathbb{N}, n \geq 1$, there
    exists a unique integer $x_0 \in \mathbb{Z}, 0 \leq x_0 < n$, such that
    $x_0 \equiv x \pmod{n}$.
\end{lemma}

\begin{definition}[reduction modulo an integer]
    \label{def:red-mod}
    For every integer $x \in \mathbb{Z}, n \in \mathbb{N}, n \geq 1$, the
    reduction of $x \pmod{n}$ is the integer $x_0$ which satisfies
    \ref{lem:red-mod}.
\end{definition}

\begin{lemma}[Properties of XOR]
    \label{lem:xor-prop}
    The XOR operation is associative, has zero as an identity, is commutative,
    and is an involution. In other words, for all $x, y, z \in \mathbb{Z}$:
    $$ \displaylines{
        (x \xor y) \xor z = x \xor (y \xor z) \cr
       x \xor 0 = x \cr
       x \xor y = y \xor x \cr
       x \xor x = 0}
   $$
\end{lemma}

\begin{lemma}[congruence modulo an integer respects XOR]
    \label{lem:xor-mod}
    For every $n \in \mathbb{N}$, congruence modulo $n$ ``respects'' XOR, in
    the sense that if $x, x, y^\prime, y^\prime \in \mathbb{Z}$ such that $x
    \equiv x^\prime \pmod{n}$ and $y \equiv y^\prime \pmod{n}$, then $$ x \xor
    y \equiv x^\prime \xor y^\prime \pmod{n} $$
\end{lemma}

This next result can be shown via Euler's Theorem (note that $\phi(2^n) =
2^{n-1})$:

\begin{definition}[even and odd (parity)]
    \label{def:parity}
    For an integer $x \in \mathbb{Z}$ to be even means that $x \equiv 0
    \pmod{2}$.

    For an integer $x \in \mathbb{Z}$ to be odd means that $x \equiv 1
    \pmod{2}$.
\end{definition}

\begin{lemma}[existence and uniqueness of a parity]
    \label{lem:parity}
    Every integer is either even or odd, but not both.
\end{lemma}

\begin{lemma}[parity of a product]
    \label{lem:product-parity}
    A product of only odd integers is odd. A product of integers is even there
    is at least one even factor.
\end{lemma}

Last but certainly not least, the next lemma is the key to the solution.

\begin{lemma}[Euler's Theorem with base=$2$]
    \label{lem:euler}
    For all integers $x, y, y^\prime \in \mathbb{Z}$, if $x$ is odd, $y,
    y^\prime \geq 0, \; y \equiv y^\prime \pmod{2^n}$, then $x^{y} \equiv
    x^{y^\prime} \pmod{2^{n+1}}$.
\end{lemma}

\subsection{Structure of \texorpdfstring{$X_n$}{Xn}}

\begin{theorem}
    \label{thm:b-even}
    Let $n \in \mathbb{N},\; n \geq 1$. For all $(a, b) \in X_n$ $b$ is even.
\end{theorem}

\begin{proof}

    Let $(a, b) \in X_n$, so by \ref{def:xn}, $0 \leq a, b < 2^n$ and $a^b
    \equiv a \xor b \pmod{2^n}$.

    By \ref{lem:mod-eqrln}, $a \equiv a \pmod{2^n}$, so by \ref{lem:xor-mod},
    $a \xor (a^b) \equiv a \xor (a \xor b) \pmod{2^n}$. By \ref{lem:xor-prop},
    $a \xor (a \xor b) = (a \xor a) \xor b = 0 \xor b = b$, which by
    substituting into the previous congruence, yields $a \xor (a^b) \equiv b
    \pmod{2^n}$.

    Also, $2^n = 0 + 2^{n-1} \cdot 2$, so $2^n \equiv 0 \pmod{2}$ by
    \ref{def:mod}, so by \ref{lem:mod-red}, $a \xor (a^b) \equiv b \pmod{2}$.

    By \ref{lem:parity}, either $a$ is even or $a$ is odd.

    \begin{itemize}

        \item

            Consider the case when $a$ is odd. Then $a^b$ is a product of
            integers with only odd factors, so $a^b$ is odd by
            \ref{lem:product-parity}. Since $a$ and $a^b$ are both odd, $a
            \equiv 1 \pmod{2}$ and $a^b \equiv 1 \pmod{2}$ by \ref{def:parity}.
            Then, by \ref{lem:xor-mod}, $a \xor a^b \equiv 1 \xor 1 \pmod{2}$.

            Finally, $a \xor (a^b) \equiv b \pmod{2}$ and $a \xor (a^b) \equiv
            1 \xor 1 \pmod{2}$, so $b \equiv 1 \xor 1 \pmod{2}$ by
            \ref{lem:mod-eqrln}. Additionally, $1 \xor 1 = 0$ by
            \ref{lem:xor-prop}, so by substituting this equation into the
            previous congruence, $b \equiv 0 \pmod{2}$, which means $b$ is even
            by \ref{def:parity}.

        \item

            Consider the case when $a$ is even. If $b$ is zero, then $b$ is
            already even. Otherwise if $b > 0$, then $a^b$ is a product of
            integers with an even factor, so $a^b$ is even by
            \ref{lem:product-parity}. Since $a$ and $a^b$ are both even, $a
            \equiv 0 \pmod{2}$ and $a^b \equiv 0 \pmod{2}$ by \ref{def:parity}.
            Then, by \ref{lem:xor-mod}, $a \xor a^b \equiv 0 \xor 0 \pmod{2}$.

            Finally, $a \xor (a^b) \equiv b \pmod{2}$ and $a \xor (a^b) \equiv
            0 \xor 0 \pmod{2}$, so $b \equiv 0 \xor 0 \pmod{2}$ by
            \ref{lem:mod-eqrln}. Additionally, $0 \xor 0 = 0$ by
            \ref{lem:xor-prop}, so by substituting this equation into the
            previous congruence, $b \equiv 0 \pmod{2}$, which means $b$ is even
            by \ref{def:parity}.

    \end{itemize}

    \hfill\proofSymbol

\end{proof}

\begin{theorem}
    \label{thm:odd-a}
    Let $n \in \mathbb{N}, n \geq 1$. For every odd $a \in \mathbb{Z},\; 0 \leq
    a < 2^n$ there exists a unique $b$ such that $(a, b) \in X_n$.
\end{theorem}

\begin{proof}

    The proof proceeds by induction on $n$.

    \begin{enumerate}

        \item

            Consider the case where $n = 1$. Let $0 \leq a < 2^1$. The only
            choices for $a$ such that $0 \leq a < 2^1$ are 0 and 1. Only 1 is
            odd. What remains is to show that there exists a unique $b \in
            \mathbb{Z}$ such that $(a, b) \in X_1$.

            Existence of $b$: By by \ref{lem:mod-eqrln}, the congruence $1
            \equiv 1 \pmod{2^1}$ holds. Additionally, $1^0 = 1$ and $1 \xor 0 =
            1$ by \ref{lem:xor-prop}. By substituting these equations into the
            congruence, $1^0 \equiv 1 \xor 0 \pmod{2^1}$. Since $0 \leq 1, 0 <
            2^1$, by \ref{def:xn} $(a, b) = (1, 0) \in X_1$.

            Uniqueness of $b$: Let $b, b^\prime$ such that $(a, b), (a,
            b^\prime) \in X_1$. Then, by \ref{def:xn}, $0 \leq b, b^\prime <
            2^1$. There are only two choices for each of $b, b^\prime$ such
            that $0 \leq b, b^\prime < 2^1$, namely 0 and 1. By
            \ref{thm:b-even} both $b$ and $b^\prime$ must be even. However, of
            the two choices, only $b = b^\prime = 0$ is even, which shows in
            particular $b = b^\prime$.

        \item

            Now suppose that \ref{thm:odd-a} holds for $n = n_0 \in \mathbb{N},
            n_0 \geq 1$. Let $0 \leq a < 2^{n_0+1}$ such that $a$ is odd. What
            remains is to show that there exists a unique $b \in \mathbb{Z}$
            such that $(a, b) \in X_{n_0+1}$.

            Existence: Let $a_0$ be the reduction of $a \pmod{2^{n_0}}$. The
            remaining proof is lengthy, so it is divided into self-contained
            steps (each one only uses the last result of the previous one):

            \begin{enumerate}

                \item

                    By the inductive hypothesis, there exists a unique $b_0$
                    such that $(a_0, b_0) \in X_{n_0}$, so $0 \leq a_0, b_0 <
                    2^{n_0}$ and $a_0^{b_0} \equiv a_0 \xor b_0 \pmod{2^{n_0}}$
                    by \ref{def:xn}. By \ref{lem:mod-eqrln}, $a_0 \equiv a_0
                    \pmod{2^{n_0}}$, so $a_0 \xor (a_0^{b_0}) \equiv a_0 \xor
                    (a_0 \xor b_0) \pmod{2^{n_0}}$ by \ref{lem:xor-mod}. By
                    \ref{lem:xor-prop}, $a_0 \xor (a_0 \xor b_0) = (a_0 \xor
                    a_0) \xor b_0 = 0 \xor b_0 = b_0$. By substituting into the
                    previous congruence, $a_0 \xor a_0^{b_0} \equiv b_0
                    \pmod{2^{n_0}}$. Note that $a_0 \equiv a \pmod{2^{n_0}}$ by
                    \ref{def:red-mod}. Therefore, $a_0^{b_0} \equiv a^{b_0}
                    \pmod{2^{n_0}}$ by \ref{lem:mod} (powers are iterated
                    products). So, by \ref{lem:xor-mod}, $a_0 \xor a_0^{b_0}
                    \equiv a \xor a^{b_0} \pmod{2^{n_0}}$. The left-hand-side
                    is congruent to $b_0 \pmod{2^{n_0}}$, so $b_0 \equiv a \xor
                    a^{b_0} \pmod{2^{n_0}}$ by \ref{lem:mod-eqrln}.

                \item

                    \label{step:odd-a-exis-b} Since $a$ is also odd, by
                    \ref{lem:euler} it follows that $a^{b_0} \equiv a^{a \xor
                    a^{b_0}} \pmod{2^{n_0+1}}$. By \ref{lem:xor-prop}, $a^{b_0}
                    ,= 0 \xor a^{b_0} = (a \xor a) \xor a^{b_0} = a \xor (a
                    \xor a^{b_0})$. Finally, substituting this last equation
                    into the congruence before it yields $a \xor (a \xor
                    a^{b_0}) \equiv a^{a \xor a^{b_0}} \pmod{2^{n_0+1}}$.
                    Almost there!

                \item

                    Let $b$ be the reduction of $a \xor a^{b_0}
                    \pmod{2^{n_0+1}}$, so $0 \leq b < 2^{n_0+1}$ and $b \equiv
                    a \xor a^{b_0} \pmod{2^{n_0+1}}$ by \ref{def:red-mod}. By
                    \ref{lem:mod-eqrln}, $a \equiv a \pmod{2^{n_0+1}}$, so $a
                    \xor (a \xor a^{b_0}) \equiv a \xor b \pmod{2^{n_0+1}}$ by
                    \ref{lem:xor-mod}. Additionally, $2^{n_0+1} = 0 + 2 \cdot
                    2^{n_0}$, so $2^{n_0+1} \equiv 0 \pmod{2^{n_0}}$ by
                    \ref{def:mod}. Thus, by \ref{lem:mod-red} $b \equiv a \xor
                    a^{b_0} \pmod{2^{n_0}}$. By \ref{lem:euler}, $a^b \equiv
                    a^{a \xor a^{b_0}} \pmod{2^{n_0+1}}$. At last, using the
                    result from step (b), by \ref{lem:mod-eqrln} $a^b \equiv
                    a^{a \xor a^{b_0}} \equiv a \xor (a \xor a^{b_0}) \equiv a
                    \xor b \pmod{2^{n_0+1}}$ impies  that $a^b \equiv a \xor b
                    \pmod{2^{n_0+1}}$. Since $0 \leq a, b < 2^{n_0+1}$, $(a, b)
                    \in X_{n_0+1}$ by \ref{def:xn}.

            \end{enumerate}

            Uniqueness of $b$: Let $b, b^\prime$ such that $(a, b), (a,
            b^\prime) \in X_{n+1}$.

            Let $a_0, b_0, b^\prime_0$ be the reductions of $a, b, b^\prime
            \pmod{2^{n_0}}$, respectively. The remaining proof is lengthy, so
            it is divided into self-contained parts (the results of both parts
            will be used afterwards):

            \begin{enumerate}

                \item

                    Since $a_0 \equiv a \pmod{2^{n_0}}$ by \ref{def:red-mod},
                    $a_0^{b_0} \equiv a^{b_0} \pmod{2^{n_0}}$ and
                    $a_0^{b^\prime_0} \equiv a^{b^\prime_0} \pmod{2^{n_0}}$ by
                    \ref{lem:mod} (powers are iterated products). 

                    Since $2^{n_0} = 0 + 2 \cdot{2^{n_0 - 1}}$, by
                    \ref{def:mod} $2^{n_0} \equiv 0 \pmod{2^{n_0 - 1}}$. Also
                    by \ref{def:red-mod} $b_0 \equiv b \pmod{2^{n_0}}$ and
                    $b^\prime_0 \equiv b^\prime \pmod{2^{n_0}}$, so then by
                    \ref{lem:mod-red}, $b_0 \equiv b \pmod{2^{n_0-1}}$ and
                    $b^\prime_0 \equiv b^\prime \pmod{2^{n_0-1}}$. Since it is
                    also true that $a$ is odd, $a^{b_0} \equiv a^b
                    \pmod{2^{n_0}}$ and $a^{b^\prime_0} \equiv a^{b^\prime}
                    \pmod{2^{n_0}}$ by \ref{lem:euler}.

                    By \ref{lem:mod-eqrln} $a_0^{b_0} \equiv a^{b_0} \equiv a^b
                    \pmod{2^{n_0}}$ implies that $a_0^{b_0} \equiv a^b
                    \pmod{2^{n_0}}$, and $a_0^{b^\prime_0} \equiv
                    a^{b^\prime_0} \equiv a^{b^\prime} \pmod{2^{n_0}}$ implies
                    that $a_0^{b^prime_0} \equiv a^b \pmod{2^{n_0}}$.

                \item
                    Since $(a, b), (a, b^\prime) \in X_{n_0+1}$, by
                    \ref{def:xn} $a^b \equiv a \xor b \pmod{2^{n_0+1}}$, and
                    $a^{b^\prime} \equiv a \xor b^\prime \pmod{2^{n_0+1}}$.
                    Since $2^{n_0+1} = 0 + 2 \cdot 2^{n_0}$, so by
                    \ref{def:mod} $2^{n_0+1} \equiv 0 \pmod{2^{n_0}}$, and by
                    \ref{lem:mod-red}, $a^b \equiv a \xor b \pmod{2^{n_0}}$,
                    and $a^{b^\prime} \equiv a \xor b^\prime \pmod{2^{n_0}}$.

                \item

                    By \ref{def:red-mod} $a_0 \equiv a \pmod{2^{n_0}}, b_0
                    \equiv b \pmod{2^{n_0}}, b^\prime_0 \equiv b^\prime
                    \pmod{2^{n_0}}$. Therefore, by \ref{lem:xor-mod}, $a_0 \xor
                    b_0 = a \xor b \pmod{2^{n_0}}$ and $a_0 \xor b^\prime_0
                    \equiv a \xor b^\prime \pmod{2^{n_0}}$.
                    
            \end{enumerate}

            In summary, the three parts above show that

            \begin{align*}
                a_0^{b_0} &\equiv a^b \pmod{2^{n_0}} & a_0^{b^\prime_0} &\equiv a^{b^\prime} \pmod{2^{n_0}} \\
                a^b &\equiv a \xor b \pmod{2^{n_0}} & a^{b^\prime} &\equiv a \xor b^\prime \pmod{2^{n_0}} \\
                a_0 \xor b_0 &\equiv a \xor b \pmod{2^{n_0}} & a_0 \xor b^\prime_0 &\equiv a \xor b^\prime \pmod{2^{n_0}}
            \end{align*}

            Therefore, using \ref{lem:mod-eqrln}, $a_0^{b_0} \equiv a_0 \xor
            b_0 \pmod{2^{n_0}}$ and $a_0^{b^\prime_0} \equiv a_0 \xor
            b^\prime_0 \pmod{2^{n_0}}$. Since it also true that $0 \leq a_0,
            b_0, b^\prime_0$, by \ref{def:xn} $(a_0, b_0), (a_0, b^\prime_0)
            \in X_{n_0}$. By the uniqueness part of the inductive hypothesis,
            it follows that $b_0 = b^\prime_0$. What remains to be shown is that
            in fact $b = b^\prime$.

            By \ref{def:red-mod}, $b \equiv b_0 \pmod{2^{n_0}}$. Substituting
            $b_0 = b^\prime_0$ yields $b \equiv b^\prime_0 \pmod{2^{n_0}}$. By
            \ref{def:red-mod}, $b^\prime_0 \equiv b^\prime \pmod{2^{n_0}}$, so
            $b \equiv b^\prime \pmod{2^{n_0}}$.

            Finally, by \ref{lem:euler}, since $a$ is odd, $a^b \equiv
            a^{b^\prime} \pmod{2^{n_0+1}}$. But since $(a, b), (a, b^\prime)
            \in X_{n_0+1}$, by \ref{def:xn} $a^b \equiv a \xor b
            \pmod{2^{n_0+1}}$ and $a^{b^\prime} \equiv a \xor b^\prime
            \pmod{2^{n_0+1}}$. Combining these three congruences by
            \ref{lem:mod-eqrln} yields $a \xor b \equiv a \xor b^\prime
            \pmod{2^{n_0}+1}$. Since $a \equiv a \pmod{2^{n_0} + 1}$ by
            \ref{lem:mod-eqrln}, by \ref{lem:xor-mod} $a \xor (a \xor b) \equiv
            a \xor (a \xor b^\prime) \pmod{2^{n_0+1}}$. By \ref{lem:xor-prop},
            $a \xor (a \xor b) = (a \xor a) \xor b = 0 \xor b = b$, and $a \xor
            (a \xor b^\prime) = (a \xor a) \xor b^\prime = 0 \xor b^\prime =
            b^\prime$. Substituting these two equations into the previous
            congruence at last gives $b \equiv b^\prime \pmod{2^{n_0+1}}$.

            By \ref{lem:red-mod}, there is a unique integer $b_0 \in
            \mathbb{Z}, 0 \leq b_0 < 2^{n_0+1}$ such that $b_0 \equiv b
            \pmod{2^{n_0+1}}$. However, $0 \leq b < 2^{n_0+1}$ and $b \equiv b
            \pmod{2^{n_0+1}}$ by \ref{lem:mod-eqrln}, but on the other hand $0
            \leq b^\prime < 2^{n_0+1}$ and $b \equiv b^\prime
            \pmod{2^{n_0+1}}$. Thus $b = b_0 = b^\prime$, so in particular, $b
            = b^\prime$.

    \end{enumerate}

    By the principle of induction, it follows that for all $n \in \mathbb{N},\;
    n \geq 1$, and for every odd $a \in \mathbb{Z},\; 0 \leq a < 2^n$ there
    exists a unique $b$ such that $(a, b) \in X_n$.

    \hfill\proofSymbol

\end{proof}

\begin{theorem}
    \label{thm:even-a}
    Let $n \in \mathbb{N}, n \geq 1$. For every $0 \leq b < 2^n$, there
    exists a unique even $a \in \mathbb{Z}$ such that $(a, b) \in X_n$.
\end{theorem}

\begin{proof}

    TODO

    \hfill\proofSymbol

\end{proof}

\subsection{The Algorithm}

TODO

\section{Proving the Preliminary Stuff}

TODO

\end{document}
